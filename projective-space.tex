\section{$4\times4$ projective space-time representations}
\label{sec:projectivespace}
The projective space-time representations in instant form dynamics were discussed in Ref.\cite{Wheeler2009}. To find the $4\times4$ projective space-time representations in the interpolation, it is convenient to start with Eq.\eqref{Jab} and the interpolation is done by a transformation matrix ($4\times4$), which is given by,
\begin{align}
    (\mathcal{R}_{\hat{a}}^{b})_{4\times4}=(\mathcal{R}_{\hat{a}}^{b})^T_{4\times4}=\begin{pmatrix}
    1&0&0&0\\
    0&1&0&0\\
    0&0&\cos{\delta}&\sin{\delta}\\
    0&0&\sin{\delta}&-\cos{\delta}
    \end{pmatrix}_{4\times4}
\end{align}
Then in interpolation form $J_{\hat{a}\hat{b}}$ becomes $J_{\hat{a}\hat{b}}=\mathcal{R}_{\hat{a}}^{~{c}}J_{cd}\mathcal{R}_{~\hat{b}}^{{d}}$, that is
\begin{align}
    J_{\hat{a}\hat{b}}&=\begin{pmatrix}
    0&-D&-\frac{\mathfrak{K}^{\hat{+}}}{\sqrt{2}}&-\frac{\mathfrak{K}^{\hat{-}}}{\sqrt{2}}\\
    D&0&\frac{P_{\hat{+}}}{\sqrt{2}}&\frac{P_{\hat{-}}}{\sqrt{2}}\\
    \frac{\mathfrak{K}^{\hat{+}}}{\sqrt{2}}&-\frac{P_{\hat{+}}}{\sqrt{2}}&0  & -{K}_{3}\\
    \frac{\mathfrak{K}^{\hat{-}}}{\sqrt{2}}&-\frac{P_{\hat{-}}}{\sqrt{2}}&{K}_{3}  & 0
  \end{pmatrix}_{4\times4}\label{Jhat+hat-},
\end{align}
where, $P_{\hat{+}}=P_0\cos{\delta}+P_3\sin{\delta}$, $P_{\hat{-}}=P_0\sin{\delta}-P_3\cos{\delta}$, $
\mathfrak{K}^{\hat{+}}=\mathfrak{K}_0\cos{\delta}+\mathfrak{K}_3\sin{\delta}$, $\mathfrak{K}^{\hat{-}}=\mathfrak{K}_0\sin{\delta}-\mathfrak{K}_3\cos{\delta}$, $D_{\hat{+}}=D\cos\delta+K_{3}\sin\delta$, $D_{\hat{-}}=D\sin\delta-K_{3}\cos\delta$, and the perpendicular components remain the same. Note that our choice of convention for $\Bar{\mathfrak{K}}_{\hat{\pm}}$ are different from $\mathfrak{K}_{\hat{\pm}}$ mentioned in section \ref{sec:conformal}, but the definitions are related as,
\begin{align}
    \begin{pmatrix}
        \mathfrak{K}^{\hat{+}}\\
        \mathfrak{K}^{\hat{-}}
    \end{pmatrix}=&\begin{pmatrix}
        \mathbb{C}&\mathbb{S}\\
        \mathbb{S}&-\mathbb{C}
    \end{pmatrix}\begin{pmatrix}
        \mathfrak{K}_{\hat{+}}\\
        \mathfrak{K}_{\hat{-}}
    \end{pmatrix}.
\end{align}
In other words, bar notation ($\Bar{\mathfrak{K}}_{\hat{\pm}}$) is an interpolating superscript ($\mathfrak{K}^{\hat{\pm}}$), i.e., $\Bar{\mathfrak{K}}_{\hat{\pm}}=\mathfrak{K}^{\hat{\pm}}=g^{\hat{\pm}\hat{\pm}}\mathfrak{K}_{\hat{\pm}}+g^{\hat{\pm}\hat{\mp}}\mathfrak{K}_{\hat{\mp}}$.


In the IFD limit $\delta\rightarrow0$ (or $\mathbb{S}\rightarrow0$ \& $\mathbb{C}\rightarrow1$), we have $P_{\hat{+}}\rightarrow P_{0}$, $P_{\hat{-}}\rightarrow -P_{3}$, $\mathfrak{K}^{\hat{+}}\rightarrow \mathfrak{K}_{0}$, $\mathfrak{K}^{\hat{-}}\rightarrow -\mathfrak{K}_{3}$, $D_{\hat{+}}\rightarrow D$, and $D_{\hat{-}}\rightarrow -K_{3}$. Likewise, in the LFD limit $\delta\rightarrow\frac{\pi}{4}$ (or $\mathbb{S}\rightarrow1$ \& $\mathbb{C}\rightarrow0$), we have $P_{\hat{\pm}}\rightarrow P_{\pm}=\frac{P_0\pm P_3}{\sqrt{2}}$, $\Bar{\mathfrak{K}}_{\hat{\pm}}\rightarrow \mathfrak{K}_{\mp}=\frac{\mathfrak{K}_0\pm \mathfrak{K}_3}{\sqrt{2}}$, and $D_{\hat{\pm}}\rightarrow D_{\pm}=\frac{D \pm K_3}{\sqrt{2}}$. Then the simplified conformal algebra in projective space-time interpolation is:
  \begin{align}
      \left[J_{{\hat{a}}{\hat{b}}}J_{{\hat{c}}{\hat{d}}}\right]=-i\left(g_{{\hat{b}}{\hat{d}}}J_{{\hat{a}}{\hat{c}}}-g_{{\hat{b}}{\hat{c}}}J_{{\hat{a}}{\hat{d}}}+g_{{\hat{a}}{\hat{c}}}J_{{\hat{b}}{\hat{d}}}-g_{{\hat{a}}{\hat{d}}}J_{{\hat{b}}{\hat{c}}}\right)\label{simplesrint}
  \end{align}
where, 
\begin{align}
    g_{\hat{a},\hat{b}}&=\begin{pmatrix}
  0&-1&0&0\\
  -1&0&0&0\\
  0&0&\mathbb{C}&\mathbb{S}\\
  0&0&\mathbb{S}&-\mathbb{C}\\
  \end{pmatrix}_{4\times4}\label{metricghat}
\end{align}
The algebra Eq.\eqref{simplesrint} with the above interpolating $4\times4$ metric will reproduce 15 explicit commutation relations in interpolation between IFD, and LFD mentioned in Table~\ref {tabel1+1interpolationlfd}.

The conformal algebra Eq.\eqref{Jhat+hat-} in interpolation form in projective-space-time implies that $J_{\hat{a}\hat{b}}$ can be written as
\begin{align}
    J_{\hat{a}\hat{b}}=i(X_{\hat{a}}\partial_{\hat{b}}-X_{\hat{b}}\partial_{\hat{a}})
\end{align}
where $\hat{a},\hat{b}\in\{-2,-1,\hat{+},\hat{-}\}$, $X_{\hat{a}}$ is the four-dimensional projective-space-time and $\partial_{\hat{a}}\equiv\frac{\partial}{\partial X^{\hat{a}}}$. With the condition that the light cone is preserved in the four-dimensional projective-space-time under the transformations generated by $ J_{\hat{a}\hat{b}}$, we write the infinitesimal conformal transformations in four-dimensional projective-space-time as
\begin{align}
    R^{\hat{a}}_{~\hat{b}}=g^{\hat{a}}_{~\hat{b}}-\frac{i}{2}\omega^{\hat{c}\hat{d}}(J_{\hat{c}\hat{d}})^{\hat{a}}_{~\hat{b}},
\end{align}
where $\omega^{\hat{a}\hat{b}}=-\omega^{\hat{b}\hat{a}}$. The generator representation $(J_{\hat{c}\hat{d}})^{\hat{a}}_{~\hat{b}}$ can be obtained by
\begin{align}
    (J_{\hat{c}\hat{d}})^{\hat{a}}_{~\hat{b}}=(J_{\hat{c}\hat{d}})^{\hat{a}\hat{f}}g_{\hat{f}\hat{b}}
\end{align}
where, $(J_{{\hat{c}}{\hat{d}}})^{{\hat{a}}{\hat{b}}}=g_{\hat{c}}^{\hat{a}} g_{\hat{d}}^{\hat{b}}-g_{\hat{c}}^{\hat{b}} g_{\hat{d}}^{\hat{a}}$. The representation matrices of conformal generators are defined by taking the first  index to be a superscript and the second subscript:
\begin{align}
    &\frac{-\mathfrak{K}^{\hat{\mu}}}{\sqrt{2}}\equiv (J_{\hat{-2}\hat{\mu}})^{\hat{a}}_{~\hat{b}}~;~~\frac{P_{\hat{\mu}}}{\sqrt{2}}\equiv (J_{\hat{-1}\hat{\mu}})^{\hat{a}}_{~\hat{b}}~;\nonumber\\
    &D_{\hat{+}}\equiv -\cos{\delta}(J_{\hat{-2}\hat{-1}})^{\hat{a}}_{~\hat{b}}-\sin{\delta}(J_{\hat{+},\hat{-}})^{\hat{a}}_{~\hat{b}}~;\nonumber\\
    &D_{\hat{-}}\equiv -\sin{\delta}(J_{\hat{-2}\hat{-1}})^{\hat{a}}_{~\hat{b}}+\cos{\delta}(J_{\hat{+},\hat{-}})^{\hat{a}}_{~\hat{b}}~.\label{implicit6x6}
\end{align}
The explicit $4\times4$ matrix representation is given in Appendix \ref{4x4}. On complex exponentiating these generators $e^{-\frac{i}{2}\omega^{\hat{c}\hat{d}}(J_{\hat{c}\hat{d}})^{\hat{a}}_{~\hat{b}}}$ then acting on $X_{\hat{a}}$, we get transformed four-dimensional projective-space-time $X^{\prime}_{\hat{a}}$. To obtain the corresponding transformations of four-dimensional space-time, we define the conformal transformations in interpolation as those transformations that preserve the light cone. This is equivalent to preserving angles, and also equivalent to preserving ratios of lengths. Let's consider a projective four-dimensional vector,
\begin{align}
    {X}_{\hat{-1}}&=\frac{-\lambda}{\sqrt{2}};\\
    {X}_{\hat{-2}}&=\frac{-\lambda}{\sqrt{2}}(x^{\hat{\mu}}.x_{\hat{\mu}});\\
     {X}_{\hat{\mu}}&=\lambda x_{\hat{\mu}},
\end{align}
which satisfies ${X}_{\hat{a}}.{X}^{\hat{a}}=0$, the dictionary connecting four-dimensional projective-space-time $X_{\hat{a}}$ and four dimensional space-time $X_{\hat{\mu}}$ reads,
\begin{align}
    x_{\hat{\mu}}&=-\frac{1}{\sqrt{2}}\frac{{X}_{\hat{\mu}}}{ {X}_{\hat{-1}}}
\end{align}
and the inverse defined as $\frac{x_{\hat{\mu}}}{x^2}=-\frac{1}{\sqrt{2}}\frac{{X}_{\hat{\mu}}}{{X}_{\hat{-2}}}$, with $x^2=\frac{{X}_{\hat{-2}}}{{X}_{\hat{-1}}}$. To find kinematic and dynamic generators of conformal transformations in the interpolation form, we calculate the transformations of the interpolating time $x^{\hat{+}\prime}$ in the Table.~\ref{tablexprime}
\begin{widetext}
\begin{center}
\begin{table}[h!]
        \centering
        \scalebox{1}{ 
        \begin{tabular}{|c|c|c|}
        \hline
             \rule{0pt}{16pt} Generators & $x^{\hat{+}\prime}$\\
             \hline
            \rule{0pt}{16pt} $\mathfrak{K}^{\hat{+}}$ & $x^{\hat{+}\prime}=\frac{x^{\hat{+}}-b^{\hat{+}}(x)^2}{1-2\mathbb{C}b^{\hat{+}}x^{\hat{+}}-2\mathbb{S}b^{\hat{+}}x^{\hat{-}}+\mathbb{C}(b^{\hat{+}})^2(x)^2}$\\
             \hline
            \rule{0pt}{16pt} $\mathfrak{K}^{\hat{-}}$ & $x^{\hat{+}\prime}=\frac{x^{\hat{+}}}{1+2\mathbb{C}b^{\hat{-}}x^{\hat{-}}-2\mathbb{S}b^{\hat{-}}x^{\hat{+}}-\mathbb{C}(b^{\hat{-}})^2(x)^2}$ \\
             \hline
             \rule{0pt}{16pt} $P_{\hat{+}}$& $x^{\hat{+}\prime}=x^{\hat{+}}+c^{\hat{+}} $\\
             \hline
             \rule{0pt}{16pt} $P_{\hat{-}}$& $x^{\hat{+}\prime}=x^{\hat{+}}$  \\
             \hline
             \rule{0pt}{16pt} $D_{\hat{+}}$ & $x^{\hat{+}\prime} = e^{-\alpha\sin{(\delta)}}\left[\left(\cosh[\alpha\cos{(\delta)}]-\mathbb{S}\sinh[\alpha\cos{(\delta)}]\right)x^{\hat{+}}+\mathbb{C}\sinh[\alpha\cos{(\delta)}] x^{\hat{-}}\right]$ \\
             \hline
             \rule{0pt}{16pt} $D_{\hat{-}}$ & $x^{\hat{+}\prime} = e^{-\alpha\cos{(\delta)}}\left[\left(\cosh[\alpha\sin{(\delta)}]+\mathbb{S} \sinh[\alpha\sin{(\delta)}]\right)x^{\hat{+}} -\mathbb{C} \sinh[\alpha\sin{(\delta)}]x^{\hat{-}}\right]$ \\
             \hline
        \end{tabular}}
        \caption{Transformation of interpolating time under each conformal generator in $1+1$}
        \label{tablexprime}
    \end{table}
    \end{center}

    \begin{center}
        \begin{table*}[t]
  \caption{\label{tab:Kinematic_and_dynamic_generators_for_different_interoplation_angles_conformal1+1}Kinematic and dynamic conformal generators for different interpolation angles in $(1+1)$}
    \begin{ruledtabular}
       \begin{tabular}{lcc}
	%\hline
	 %\hline
	Interpolation angle & Kinematic & Dynamic \\
	\hline
	\rule{0pt}{3ex} $\delta=0$ & $P_{3}$, $D$ & $ K_{3}$, $P_{0}$, $\mathfrak{K}_{{0}}$, $\mathfrak{K}_{{3}}$\\
	$0<\delta<\pi/4$ &  $P_{\mT}$, $D=\left(D_{\hat{+}}c+D_{\hat{-}}s\right)$ &  $K_3=\left(D_{\hat{+}}s-D_{\hat{-}}c\right)$,  $P_{\pT}$, $\mathfrak{K}^{\hat{+}}$, $\mathfrak{K}^{\hat{-}}$\\
	$\delta=\pi/4$ & $P_{-}$, $D_{+}=\frac{(D+K_3)}{\sqrt{2}} $, $D_{-}=\frac{(D-K_3)}{\sqrt{2}}$, $\mathfrak{K}_{{-}}$& $P_{+}$, $\mathfrak{K}_{{+}}$\\
	%\hline
	 %\hline
      \end{tabular}
    \end{ruledtabular}
\end{table*}
    \end{center}
\end{widetext}

\begin{widetext}
\begin{center}
\begin{table}[!htb]
        \begin{minipage}{.5\linewidth}
        \begin{tabular}{|c|c|c|}
        \hline
             \rule{0pt}{16pt} Generators & $x^{{0}\prime}$\\
             \hline
            \rule{0pt}{16pt} $\mathfrak{K}_{{0}}$ & $x^{{0}\prime}=\frac{x^{{0}}-b^{{0}}(x)^2}{1-2b^{{0}}x^{{0}}+(b^{{0}})^2(x)^2}$\\
             \hline
            \rule{0pt}{16pt} $-\mathfrak{K}_{{3}}$ & $x^{{0}\prime}=\frac{x^{{0}}}{1+2b^{{3}}x^{{3}}+(b^{{3}})^2(x)^2}$ \\
             \hline
             \rule{0pt}{16pt} $P_{{0}}$& $x^{{0}\prime}=x^{{0}}+c^{{0}} $\\
             \hline
             \rule{0pt}{16pt} $-P_{{3}}$& $x^{{0}\prime}=x^{{0}}$  \\
             \hline
             \rule{0pt}{16pt} $D$ & $x^{{0}\prime} = e^{-\alpha}x^{0}$ \\
             \hline
             \rule{0pt}{16pt} $-K_{3}$ & $x^{{0}\prime} = \cosh[\alpha]x^{0}-\sinh[\alpha] x^{3}$ \\
             \hline
        \end{tabular}
        \caption{Transformation of instant form time in $1+1$}
        \label{tablex0prime}
        \end{minipage}%
    \begin{minipage}{.5\linewidth}
        \begin{tabular}{|c|c|c|}
        \hline
             \rule{0pt}{16pt} Generators & $x^{{+}\prime}$\\
             \hline
            \rule{0pt}{16pt} $\mathfrak{K}_{{-}}$ & $x^{{+}\prime}=x^{+}$\\
             \hline
            \rule{0pt}{16pt} $\mathfrak{K}_{{+}}$ & $x^{{+}\prime}=\frac{x^{{+}}}{1-2b^{{-}}x^{{+}}}$ \\
             \hline
             \rule{0pt}{16pt} $P_{{+}}$& $x^{{+}\prime}=x^{{+}}+c^{{+}} $\\
             \hline
             \rule{0pt}{16pt} $P_{{-}}$& $x^{{+}\prime}=x^{{+}}$  \\
             \hline
             \rule{0pt}{16pt} $D_{{+}}$ & $x^{{+}\prime} = e^{-\sqrt{2}\alpha}x^{{+}}$ \\
             \hline
             \rule{0pt}{16pt} $D_{{-}}$ & $x^{{+}\prime} = x^{{+}}$ \\
             \hline
        \end{tabular}
        \caption{Transformation of light-front time in $1+1$}
        \label{tablexplusprime}
          \end{minipage} 
    \end{table}
    \end{center}
\end{widetext}


The Table.~\ref{tablexprime} shows the transformations of the interpolating time $x^{\hat{+}\prime}$. The IFD limit $\mathbb{C}=1;~\mathbb{S}=0$ is given the Table.~\ref{tablex0prime} and the LFD limit $\mathbb{C}=0;~\mathbb{S}=1$ is given the Table.~\ref{tablexplusprime}

%We note that The transformations further simplifies for $\mathfrak{K}_{\pm}$ in LFD limit, $x^{+}$ transforms as $ x^{+\prime} =x^{+}$ under $\mathfrak{K}^{-}$.

Among the six conformal generators, the two generators ($P_{\mT}, D$) are always kinematic in the sense that the $x^{\pT}=0$ plane is intact under the transformations generated by them. Though dilation $D$ is always kinematic, but it's combination with $K_3$ which is $D_{\hat{\pm}}$ is is dynamic in between IFD and LFD, in IFD limit $D_{\hat{+}}$ is kinematic, $D_{\hat{-}}$ is dynamic, in LFD limit $D_{\hat{\pm}}$ is kinematic. The transformations further simplifies for $\mathfrak{K}_{\pm}$ in LFD limit, $x^{+}$ transforms as $ x^{+\prime} =x^{+}$ under $\mathfrak{K}_{-}$. We get one more kinematic generator in $(1+1)$. The set of kinematic and dynamic conformal generators in (1+1) depending on the interpolation angle is summarized in Table~\ref {tab:Kinematic_and_dynamic_generators_for_different_interoplation_angles_conformal1+1}. 

Dirac\cite{Dirac1936} has shown the existence of an isomorphism between the $SO(3+1,2)$ conformal group and Dirac matrices. Later, Hepner\cite{Hepner1962} has explicitly indicated the isomorphism between the group of Dirac's four-row $\gamma$-matrices and the continuous conformal group in Euclidean space. Our future will extend the isomorphism between the conformal group Eq.\eqref{Jhat+hat-} and the group of fifteen matrices of the $\gamma$'s and their products in the interpolation projective-space-time

